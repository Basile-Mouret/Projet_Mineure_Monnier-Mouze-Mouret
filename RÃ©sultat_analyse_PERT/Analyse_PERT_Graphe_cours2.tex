\documentclass{article}
\usepackage[utf8]{inputenc}
\usepackage[pdf]{graphviz}
\usepackage[autosize]{dot2texi}
\usepackage{tikz}
\usetikzlibrary{shapes,arrows}


\title{Analyse de votre projet}
\begin{document}
\maketitle


\section{Faisabilite du projet}
 Votre projet ne contient pas de cycle, il est faisable.
Nous pouvons donc l'analyser et vous aider à organiser votre travail.

\section{Visualisation du projet par un graphe}
\subsection{Graphe image}
Les chemins critiques sont marques en rouge dans le graphe.
\begin{dot2tex}[options=-tmath,scale=0.2916666666666667]digraph grours {rankdir=LR;
DP [label = "DP,7"]; PE [label = "PE,3"]; R1 [label = "R1,2"]; CM [label = "CM,10"]; R [label = "R,0"]; C [label = "C,7"]; DRC [label = "DRC,7"]; MP [label = "MP,14"]; GE1 [label = "GE1,3"]; T [label = "T,14"]; F [label = "F,7"]; IE [label = "IE,3"]; C1 [label = "C1,7"]; S [label = "S,3"]; PP [label = "PP,7"]; GE2 [label = "GE2,3"]; IS [label = "IS,7"]; PC [label = "PC,60"]; P [label = "P,7"]; FC [label = "FC,7"]; IW [label = "IW,3"]; PM [label = "PM,7"]; FE [label = "FE,7"]; IC [label = "IC,7"]; DP -> GE2[label = "7.0",color = "red"];DP -> C[label = "7.0"];CM -> FC[label = "10.0",color = "red"];CM -> S[label = "10.0"];C -> PE[label = "7.0"];C -> C1[label = "7.0"];C -> P[label = "7.0"];C -> R1[label = "7.0"];DRC -> MP[label = "7.0",color = "red"];MP -> DP[label = "14.0",color = "red"];MP -> R[label = "14.0"];GE1 -> DRC[label = "3.0",color = "red"];T -> FE[label = "14.0",color = "red"];T -> PE[label = "14.0"];F -> GE1[label = "7.0",color = "red"];C1 -> CM[label = "7.0"];S -> R1[label = "3.0"];S -> R[label = "3.0"];PP -> PM[label = "7.0",color = "red"];GE2 -> T[label = "3.0",color = "red"];GE2 -> IE[label = "3.0"];IS -> PP[label = "7.0",color = "red"];PC -> F[label = "60.0",color = "red"];P -> CM[label = "7.0"];P -> S[label = "7.0"];FC -> IC[label = "7.0",color = "red"];FC -> IW[label = "7.0"];FC -> IS[label = "7.0",color = "red"];IW -> PP[label = "3.0"];FE -> CM[label = "7.0",color = "red"];IC -> PP[label = "7.0",color = "red"];}
\end{dot2tex}
\subsection{Description des taches}PC : Permis de construire \newline{}F : Fondations\newline{}GE1 : Passage des gaines et évacuations \newline{}DRC : Dalle rez de chaussée\newline{}MP : Murs porteurs \newline{}DP : Dalles plafond \newline{}C : Chape\newline{}GE2 : Passage gaines et évacuation\newline{}T : Toiture\newline{}IE : Installation électrique et évacuation\newline{}C1 : Carrelage du sol\newline{}P : Parquets\newline{}PE : Portes extèrieures\newline{}FE : Fenêtres\newline{}CM : Cloisons et menuiserie intérieure\newline{}S : Serrurerie\newline{}FC : Finition des cloisons\newline{}R1 : Revêtement des sols (moquettes) \newline{}R : Réception de la maison \newline{}IS : Implantation des salles de bain\newline{}IW : Implantation des wc\newline{}IC : Implantation de la cuisine \newline{}PP : Peinture des plafonds \newline{}PM : Peinture des murs\newline{}
\subsection{Tableau des dates}

Ce tableau montre les dates auxquelles vous pourrez commencer chaque tâche au plus tôt et les dates pour lesquelles elles devront êtres finies pour ne pas retarder l'ensemble du projet
Si la tache est de couleur rouge, elle est désignée comme critique,elle n'a pas de mou, c'est a dire qu'un retard sur cette tache implique un retard sur l'ensemble du Projet.\\

\begin{tabular}{|l|l|l|l|l|}
\hline 
Tache & Duree & Date de debut au plus tot & Date de fin au plus tard & Mou\tabularnewline
\hline

PC&60.0&0&60.0&\textcolor{red}{critique}\tabularnewline
\hline
F&7.0&60.0&67.0&\textcolor{red}{critique}\tabularnewline
\hline
GE1&3.0&67.0&70.0&\textcolor{red}{critique}\tabularnewline
\hline
DRC&7.0&70.0&77.0&\textcolor{red}{critique}\tabularnewline
\hline
MP&14.0&77.0&91.0&\textcolor{red}{critique}\tabularnewline
\hline
DP&7.0&91.0&98.0&\textcolor{red}{critique}\tabularnewline
\hline
C&7.0&98.0&115.0&10.0\tabularnewline
\hline
GE2&3.0&98.0&101.0&\textcolor{red}{critique}\tabularnewline
\hline
T&14.0&101.0&115.0&\textcolor{red}{critique}\tabularnewline
\hline
IE&3.0&101.0&160.0&56.0\tabularnewline
\hline
C1&7.0&105.0&122.0&10.0\tabularnewline
\hline
P&7.0&105.0&122.0&10.0\tabularnewline
\hline
PE&3.0&115.0&160.0&42.0\tabularnewline
\hline
FE&7.0&115.0&122.0&\textcolor{red}{critique}\tabularnewline
\hline
CM&10.0&122.0&132.0&\textcolor{red}{critique}\tabularnewline
\hline
S&3.0&132.0&158.0&23.0\tabularnewline
\hline
FC&7.0&132.0&139.0&\textcolor{red}{critique}\tabularnewline
\hline
R1&2.0&135.0&160.0&23.0\tabularnewline
\hline
R&0.5&135.0&160.0&24.5\tabularnewline
\hline
IS&7.0&139.0&146.0&\textcolor{red}{critique}\tabularnewline
\hline
IW&3.0&139.0&146.0&4.0\tabularnewline
\hline
IC&7.0&139.0&146.0&\textcolor{red}{critique}\tabularnewline
\hline
PP&7.0&146.0&153.0&\textcolor{red}{critique}\tabularnewline
\hline
PM&7.0&153.0&160.0&\textcolor{red}{critique}\tabularnewline
\hline

\end{tabular}
La duree totale de votre projet est de 160.0.

\end{document}