\documentclass{article}
\usepackage[utf8]{inputenc}
\usepackage[pdf]{graphviz}
\usepackage[autosize]{dot2texi}
\usepackage{tikz}
\usetikzlibrary{shapes,arrows}


\title{Analyse de votre projet}
\begin{document}
\maketitle


\section{Faisabilite du projet}
 Votre projet ne contient pas de cycle, il est faisable.
Nous pouvons donc l'analyser et vous aider à organiser votre travail.

\section{Visualisation du projet par un graphe}
\subsection{Graphe image}
Les chemins critiques sont marques en rouge dans le graphe.
\begin{dot2tex}[options=-tmath,scale=1.0]digraph grours {rankdir=LR;
E [label = "E,9"]; F [label = "F,6"]; A [label = "A,7"]; B [label = "B,9"]; G [label = "G,5"]; D [label = "D,8"]; C [label = "C,12"]; E -> F[label = "9.0",color = "red"];E -> G[label = "9.0"];A -> C[label = "7.0"];A -> D[label = "7.0"];B -> D[label = "9.0",color = "red"];D -> E[label = "8.0",color = "red"];C -> F[label = "12.0"];}
\end{dot2tex}
\subsection{Description des taches}A : a\newline{}B : b\newline{}C : c\newline{}D : d\newline{}E : e\newline{}F : f\newline{}G : g\newline{}
\subsection{Tableau des dates}

Ce tableau montre les dates auxquelles vous pourrez commencer chaque tâche au plus tôt et les dates pour lesquelles elles devront êtres finies pour ne pas retarder l'ensemble du projet
Si la tache est de couleur rouge, elle est désignée comme critique,elle n'a pas de mou, c'est a dire qu'un retard sur cette tache implique un retard sur l'ensemble du Projet.\\

\begin{tabular}{|l|l|l|l|l|}
\hline 
Tache & Duree & Date de debut au plus tot & Date de fin au plus tard & Mou\tabularnewline
\hline

A&7.0&0&9.0&2.0\tabularnewline
\hline
B&9.0&0&9.0&\textcolor{red}{critique}\tabularnewline
\hline
C&12.0&7.0&26.0&7.0\tabularnewline
\hline
D&8.0&9.0&17.0&\textcolor{red}{critique}\tabularnewline
\hline
E&9.0&17.0&26.0&\textcolor{red}{critique}\tabularnewline
\hline
F&6.0&26.0&32.0&\textcolor{red}{critique}\tabularnewline
\hline
G&5.0&26.0&32.0&1.0\tabularnewline
\hline

\end{tabular}
La duree totale de votre projet est de 32.0.

\end{document}