\documentclass{article}
\usepackage[utf8]{inputenc}
\usepackage[pdf]{graphviz}
\usepackage[autosize]{dot2texi}
\usepackage{tikz}
\usetikzlibrary{shapes,arrows}


\title{Analyse de votre projet}
\begin{document}
\maketitle


\section{Faisabilite du projet}
 Votre projet ne contient pas de cycle, il est faisable.
\section{Visualisation du projet par un graphe}
\subsection{Graphe image}
Les chemins critiques sont marques en rouge dans le graphe.
\begin{dot2tex}[options=-tmath,scale=1.0]digraph grours {rankdir=LR;
E [label = "E,9"]; G [label = "G,5"]; B [label = "B,9"]; D [label = "D,8"]; A [label = "A,7"]; F [label = "F,6"]; C [label = "C,12"]; E -> F[label = "9.0",color = "red"];E -> G[label = "9.0"];B -> D[label = "9.0",color = "red"];D -> E[label = "8.0",color = "red"];A -> C[label = "7.0"];A -> D[label = "7.0"];C -> F[label = "12.0"];}
\end{dot2tex}
\subsection{Tableau des dates}
\newline{}
\begin{tabular}{|l|M|N|C|}
\hline 
Tache & Date de debut au plus tot & Date de fin au plus tard & Criticalite\tabularnewline
\hline

B&0&9.0&critique\tabularnewline
\hline
A&0&9.0&non critique\tabularnewline
\hline
C&7.0&26.0&non critique\tabularnewline
\hline
D&9.0&17.0&critique\tabularnewline
\hline
E&17.0&26.0&critique\tabularnewline
\hline
G&26.0&32.0&non critique\tabularnewline
\hline
F&26.0&32.0&critique\tabularnewline
\hline

\end{tabular}
\end{document}