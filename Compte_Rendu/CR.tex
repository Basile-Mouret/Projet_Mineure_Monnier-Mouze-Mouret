\documentclass{article}
\usepackage{graphicx} % Required for inserting images
\usepackage{pythonhighlight}
\usepackage{amsmath,amssymb,enumerate,graphicx,pgf,tikz,fancyhdr}
\usetikzlibrary{trees,arrows}
\usetikzlibrary{backgrounds}
\usepackage[T1]{fontenc}
\usepackage[french]{babel}
\usepackage{geometry}
\usepackage{tabvar}
\geometry{hmargin=2.2cm,vmargin=1.5cm}

\usetikzlibrary {arrows.meta,positioning}



\title{TP 1: Parcours d'Arbre Binaire.}
\author{Guillermo Mouze, Benjamin Monnier, Basile Mouret}
\date{Octobre 2023}
\renewcommand{\contentsname}{Table des Matières}

\begin{document}

\maketitle
\tableofcontents{}

\section{Introduction}

\section{Classe Graphe}
Nous avons codé la classe graphe lors du troisième TP de mineure.
Pour cee Projet nous réutilision la classe graphe crée lors du TP de mineurs. Nous l'adaptons à
notre probleme.


\section{Convertir fichier  en graphe}
convertir en fichier 'csv',récupérer les infos interessantes dans le fichier csv pour creer un arbre, 
necessité de former tout les arcs d'un certain noeuds-> creer une fonction couple


section{Problématqique}
On consid`ere un projet compos ́e de tˆaches auxquelles on associe une dur ́ee d’ex ́ecution et des relations avec
d’autres tˆaches. Chaque tˆache n ́ecessite que les tˆaches qui la pr ́ec`edent aient  ́et ́e ex ́ecut ́ees pour pouvoir d ́ebuter.
Le but du projet est de cr ́eer des structures de donn ́ees et des programmes permettant d’ ́evaluer la faisabilit ́e des
projets et les conditions en termes de temporalit ́e.
2 Architecture du programme
Notre programme est divis ́e en deux fichiers. Le fichier f onctions principales.py d ́efinit la classe Graphe
et contient toutes les fonctions sauf celle du rendu de l’analyse qui, elle, est contenue dans le fichier analyse.py.
2.1 Documentation des fonctions
— contient cycle : permet de d ́etecter si un graphe contient des cycles.
Prend en entr ́ee un graphe et renvoie un bool ́een.
— tous chemins : renvoie une liste contenant tous les chemins possibles entre deux points. Est n ́ecessaire
pour d ́eterminer les chemins critiques.
Prend en entr ́ee un graphe, les points de d ́epart et d’arriv ́ee et des listes contenant les noeuds visit ́es, le
chemin parcouru et l’ensemble des chemins parcourus. Ces trois listes sont n ́ecessaires car la fonction est
r ́ecursive.
— Afficher tous chemins : permet de renvoyer le r ́esultat de tous chemins sans avoir `a sp ́ecifier les listes
vides.
Prend en entr ́ee un graphe et les points de d ́epart et d’arriv ́ee.
— lire taches : permet de placer les donn ́ees contenues dans le fichier de tˆaches dans une liste. Prend en
entr ́ee le fichier contenant les tˆaches et leurs informations et renvoie une liste de tˆaches. Ces tˆaches sont
repr ́esent ́ees par une liste contenant dans l’ordre suivant : identifiant de la tˆache, dur ́ee de la tˆache, des-
cription de la tˆache.
— lire liaisons : permet de placer les donn ́ees contenues dans le fichier de liaisons dans un ensemble.
Prend en entr ́ee le fichier contenant les liaisons et renvoie un ensemble de couples de la forme (point de
d ́epart , point d’arriv ́ee)
— duree taches : cr ́ee un dictionnaire associant `a chaque tˆaches sa dur ́ee.
Prend en entr ́ee le fichier de tˆaches et renvoie le dictionnaire.
— generer graphe : cr ́ee le graphe associ ́e au syst`eme de tˆaches contenu dans les fichiers fournis.
Prend en entr ́ee les fichiers de tˆaches et de liaisons et renvoie le graphe associ ́e.
— arc present : permet de savoir s’il existe un chemin entre deux points.
Prend en entr ́ee un graphe et deux points et renvoie un bool ́een disant si oui ou non il y a un chemin entre
ces deux points.
— parcoursProfondeur : Renvoie le parcours en profondeur d’un graphe sous forme de liste.
— chemin critique : trouve les chemins critiques, c’est-`a-dire les chemins contenant les points critiques (les
points dont le retard entraˆınera un retard global du projet), d’un graphe.
Prend en entr ́ee un graphe, un point de d ́epart, un point d’arriv ́ee et le fichier de tˆaches relatif au graphe.
Renvoie un doublet form ́e par la liste du/des chemin/s critique/s et de la dur ́ee critique.
— triTopologique : renvoie le tri topologique des noeuds d’un graphe, c’est-`a-dire tous les noeuds du graphe
tri ́es du d ́ebut `a la fin, par niveaux.
1
Prend en entr ́ee le graphe.
— date au plus tot : calcule la date de d ́epart possible au plus t ˆot pour chaque tˆaches.
Prend en entr ́ee un graphe et le fichier de tˆache associ ́e. Retourne un dictionnaire associant `a chaque tˆache
sa date de commencement au plus t ˆot.
— date au plus tard : calcule la date de d ́epart au plus tard pour chaque tˆache sous peine d’engendrer un
retard global du projet. Prend en entr ́ee un graphe et le fichier de tˆaches associ ́e. Retourne un dictionnaire
associant `a chaque tˆache sa date de d ́epart au plus tard.
— rediger rapport : cr ́ee un fichier LATEXcontenant les analyses demand ́ees `a l’aide des programmes pr ́ec ́edents.
Prend en entr ́ee les fichiers de liaison et de tˆache d’un p
\p
\end{document} 