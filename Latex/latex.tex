\documentclass{article}
\usepackage{graphicx} % Required for inserting images
\usepackage[french]{babel}
\usepackage[utf8]{inputenc}
\usepackage{listings}
\usepackage{amsmath,amssymb}           
\usepackage[T1]{fontenc}
\usepackage{graphicx}
\usepackage{tikz}
\usepackage{pythonhighlight}
\title{\bf \Huge Manuel d'utilisateur du logiciel PERT}
\author{Benjamin Monnier\\ Basile Mourret\\ Guillermo Mouze}
\date{}
\begin{document}
\maketitle
\centerline{\includegraphics{3.png}}
\section{La méthode PERT}
La méthode PERT permet d’organiser les tâches de votre projet en fonction de leurs durées respectives et de
leurs dépendances.
\\ 
Ce dernier identifie donc les tâches qui ne doivent souffrir d’aucun retard sous peine de retarder l’ensemble du projet.
\\\\
Le PERT est aussi utilisé pour suivre le déroulement de votre projet. On
peut ainsi reporter les avances et les retards de chaque tâche pour déteminer les changements nécessaire dans la programmation des dates. 


\section{Format des données}

Le programme nécéssite un fichier d'entrée comportant les données de type 

\textit{.csv}.
\\
Le fichier devra respecter rigoureusement le format ci-dessous.
Les durées des différentes tâches seront exprimées en jour.
Chaque tâche sera indiqué sous la forme:


\begin{flushleft}
    \bf
Identificateur,Description,Durée,Précédente(s)
\end{flushleft}
\\
\underline{EXEMPLE:}
\\\\
Le passage des gaines et évacuations (GE1), dont la durée est 3 jours, nécessitant des Fondations (F) au préalable est notée:
\begin{flushleft}
    \bf
  GE1,Passage des gaines et évacuations ,3,F
\end{flushleft}

Les tâches seront listées sans saut de ligne mais simplement avec un simple retour à la ligne entre elle.
\\
Voici un exemple de fichier \textit{.csv} accepté:
\\\\

\centerline{\includegraphics[height=100]{1.png}}
\\\\

\section{Comment Procéder}
Creer votre fichier \textit{tache.csv} avec les tâches de votre projet et placer le fichier \textit{PERT.exe} et votre fichier \textit{tache.csv} dans un même dossier.
\\
Lancer le fichier \textit{PERT.exe} et rentrer dans le terminal \textit{tache.csv}.
\\
\begin{flushleft}
    \bf
Le Tour est joué!
\end{flushleft}
Le programme a créé un fichier
LATEX nommé \textit{Rendu.tex} contenant un résumé de votre projet : 
\\
— analyse de faisabilité de l’ordonnancement des tâches.
\\
— Les chemins critiques seront explicités sur la visualisation du graphe de tâches.
\\
— les comptes rendu d’exécution.
\\
— l’historique du projet.
\\
Pour l'avoir sous format PDF veuillez compiler le code LateX sur le site Overleaf.com.


\end{document}
